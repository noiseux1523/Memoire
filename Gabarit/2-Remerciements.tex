% Remerciements
%
%   Grâce aux remerciements, l'auteur attire l'attention du lecteur
% sur l'aide que certaines personnes lui ont apportée, sur leurs
% conseils ou sur toute autre forme de contribution lors de la
% réalisation de son mémoire. Le cas échéant, c'est dans cette section
% que le candidat doit témoigner sa reconnaissance à son directeur de
% recherche, aux organismes dispensateurs de subventions ou aux
% entreprises qui lui ont accordé des bourses ou des fonds de
% recherche.
\chapter*{ACKNOWLEDGEMENTS}\thispagestyle{headings}
\addcontentsline{toc}{compteur}{ACKNOWLEDGEMENTS}
%
Texte.
FACULTATIF