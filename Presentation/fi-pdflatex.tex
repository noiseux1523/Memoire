%%%%%%%%%%%%%%%%%%%%%%%%%%%%%%%%%%%%%%%%%%%%%%%%%%%%%%%%%%%%%%%%%%%%
%% I, the copyright holder of this work, release this work into the
%% public domain. This applies worldwide. In some countries this may
%% not be legally possible; if so: I grant anyone the right to use
%% this work for any purpose, without any conditions, unless such
%% conditions are required by law.
%%%%%%%%%%%%%%%%%%%%%%%%%%%%%%%%%%%%%%%%%%%%%%%%%%%%%%%%%%%%%%%%%%%%

\documentclass{beamer}
\usetheme[faculty=fi]{fibeamer}
\usepackage{framed}
\usepackage[utf8]{inputenc}
\usepackage{listings}
\usepackage{color}

\definecolor{dkgreen}{rgb}{0,0.6,0}
\definecolor{gray}{rgb}{0.5,0.5,0.5}
\definecolor{mauve}{rgb}{0.58,0,0.82}

\lstset{frame=tb,
	language=Java,
	aboveskip=3mm,
	belowskip=3mm,
	showstringspaces=false,
	columns=flexible,
	basicstyle={\small\ttfamily},
	numbers=none,
	numberstyle=\tiny\color{gray},
	keywordstyle=\color{blue},
	commentstyle=\color{dkgreen},
	stringstyle=\color{mauve},
	breaklines=true,
	breakatwhitespace=true,
	tabsize=3
}
\usepackage[
  main=english, %% By using `czech` or `slovak` as the main locale
                %% instead of `english`, you can typeset the
                %% presentation in either Czech or Slovak,
                %% respectively.
  czech, slovak %% The additional keys allow foreign texts to be
]{babel}        %% typeset as follows:
%%
%%   \begin{otherlanguage}{czech}   ... \end{otherlanguage}
%%   \begin{otherlanguage}{slovak}  ... \end{otherlanguage}
%%
%% These macros specify information about the presentation
\title{
RECOMMENDING WHEN DESIGN TECHNICAL DEBT SHOULD BE SELF-ADMITTED} %% that will be typeset on the
\subtitle{Computer Engineering} %% title page.
\author{Cedric Noiseux}
\date{December 5, 2017}
\defbeamertemplate*{title page}{customized}
{
	\centering
	\begin{framed}
		\usebeamerfont{title}\inserttitle\par
	\end{framed}
	\usebeamerfont{subtitle}\textcolor{white}{\insertsubtitle}\par
	\begin{figure}[t]
		\centering
		\includegraphics[width=45mm]{resources/logo}
		\vspace{-4mm}
	\end{figure}
	\usebeamerfont{author}\textcolor{white}{\insertauthor}\par
	%\usebeamerfont{institute}\textcolor{lightgray}{\insertinstitute}\par
	\usebeamerfont{date}\textcolor{white}{\insertdate}\par
}
%% These additional packages are used within the document:
\usepackage{ragged2e}  % `\justifying` text
\usepackage{booktabs}  % Tables
\usepackage{tabularx}
\usepackage{tikz}      % Diagrams
\usetikzlibrary{calc, shapes, backgrounds}
\usepackage{amsmath, amssymb}
\usepackage{url}       % `\url`s
\usepackage{listings}  % Code listings
\frenchspacing
\begin{document}
	\begin{darkframes}
		%%%%%%%%%
		% Title Page %
		%%%%%%%%%
		
		\frame{\maketitle}
  
  		%%%%%%%%%%%%%
  		% Table of Content %
  		%%%%%%%%%%%%%
  		
				\begin{frame}[allowframebreaks]
					\frametitle{Table of Contents}
					\tableofcontents[sections={1-2}]
						\framebreak
					\tableofcontents[sections={3-4}]
				\end{frame}
	
		\section{INTRODUCTION}
		
			%%%%%%%%%%%%%%%%%%
			% Concepts and Definitions %
			%%%%%%%%%%%%%%%%%%
	
			\subsection{Concepts and Definitions}	  
				\begin{frame}{Concepts and Definitions}		
					\begin{block}{Technical Debts}
						They are not quite right code which we postpone making it right~\cite{cunn92}. They are temporary and suboptimal solutions or workarounds.
					\end{block}
					\begin{block}{Reasons}
						Quickly fix an issue\\
						Early conception stages\\
						Lack of comprehension, skill or experience	
					\end{block}
				\end{frame}
						
				\begin{frame}{Concepts and Definitions}	
					\begin{block}{Types}
						Design\\
						Requirement\\
						Code\\
						Test\\
						Documentation
					\end{block}	
				\end{frame}
			
			
			Technical Debts (defintion, impact)
			Types of Debts
			Self-Admitted Technical Debts (definition, exploratory study)
						
			\subsection{Elements of the Problematic}
			
			Low prediction (many FP)
			Inefficient strategies (refactoring, tracking, etc)
			Ineficient approaches (pattern matching, ML)
			Research questions (main and 3 sub-questions)
	
			\subsection{Objectives}
			
			Research objective (main and 2 application scenarios)
			Design objectives (5 objectives)

	    \section{TEDIOUS}
	    
	    	\subsection{Study Definition}	
	    	
	    	9 OOS + Java
	    	More method-related than class-related
		    Unbalance
		    
		    \subsection{The Approach}
		    
		    Method-level
		    Image du process (separer en differents process et expliquer)
		    	- Training and test sets
		    	- SATD matching
		    	- Features
		    	- Preprocessing
		    	- Building and applying machine learners (training and testing)
		    
		    \subsection{Study Results}
		    
		    Analysis Method
		    Within project performance
		    Cross project performance
		    TEDIOUS vs method level
		    Qualitative analysis
		    
	\end{darkframes}

	\section{CONVOLUTIONAL NEURAL NETWORK}
	
		\subsection{Introduction}
		
		Inspirations (Kim, Dos Santos)
		Description

		\subsection{The Approach}
		
		Method-level
		Image du process (separer en differents process et expliquer)
		- Training and test sets
		- SATD matching
		- Source code
		- Preprocessing/Word Embedings
		- Building and applying machine learners (training and testing)

		\subsection{Study Definition}	
		
		Research questions (main and 3 sub-questions)
		9 OOS + Java
		More method-related than class-related
		Unbalance

		\subsection{Study Results}
		
		Source code comments only
		Source code with comments
		Source code without comments
		Source code partially with comments
			
	\section{CONCLUSION}
	
		\subsection{Threats to Validity}
		
		Construct
		Internal
		Conclusion
		Reliability
		External

		\subsection{Limitations}	
		
		Number of metrics/warning
		Number of LOC
		Process inherent errors
		Default configurations (no real optimizations)
		Generalization of results
		Only in Java
		Performance evaluation can be better for CNN
		
		\subsection{Summary and Future Work}	
	
		TEDIOUS approach
		CNN approach
		Dataset
		TEDIOUS results
		CNN results
		Applicability scenarios
			- Recommendation system
			- Complement to smell detectors
		Improvements
			- Optimization
			- Pattern matching process
			- More examples to train
			- More positive examples
			- More metrics/warnings and source code
			- Other TD types
			- Extend to other languages and domains
			
		\begin{frame}[label=bibliography]{Bibliography}
		\framesubtitle{\TeX, \LaTeX, and Beamer}
		\begin{thebibliography}{9}
			\bibitem{cunn92}
			W~Cunningham.
			\emph{The wycash portfolio management system}.
			dans Addendum to the Proceedings on Object-oriented Programming Systems, Languages, and Applications (Addendum), s\'{e}rie OOPSLA ’92, New York, NY, USA: ACM, 1992, pp 29–30, DOI: 10.1145/157709.157715. En ligne: http://doi.acm.org/10.1145/157709.157715.
			\end{thebibliography}
		\end{frame}
	
\end{document}
